\documentclass{article}
%% Layout packages
\usepackage{geometry}

%% Adjust the margin
\geometry{a4paper}
\setlength{\voffset}{-0.75in}

%% Remove the page numbering
\pagestyle{empty}

\begin{document}
\textbf{Extension of the compressed sensing based medical ultrasound imaging demonstrator to doppler flow imaging}

\vspace{\baselineskip}
\par Ultrasound imaging is a common medical imaging modality nowadays. The major problem is the amount of data and calculations needed to reconstruct an image. In order to address this problem, LTS5 have developed a compressed beamforming algorithm which is able to recover high quality images from less than 30\% of the data required by classical methods.
\par Doppler imaging is a widely used modality of ultrasound imaging, taking advantage of the Doppler effect, in order to assess and quantify whether structures (usually blood) are moving. This procedure is mainly used in echocardiography (heart and blood vessels), allowing us to determine the velocity and direction of blood flow. In order to accurately measure the complex blood flow in the human body, fast acquisition rates are required. Hence, ultrafast ultrasound is a promising technique but implies massive computations and large amount of data. Compressed sensing approaches can relax the amount of data required.

\vspace{\baselineskip}
The main goal of this project is to extend the experimental demonstrator with a Doppler ultrasound method compliant with the compressed sensing (CS) based framework for ultrafast ultrasound imaging developed at LTS5.

\vspace{\baselineskip} 
\textbf{Objectives:}
\begin{itemize}
	\item Familiarize with the CS based framework for ultrafast ultrasound imaging developed at LTS5.
	\item Familiarize with the current demonstrator based on the ULA-OP system.
	\item Study the state-of-the-art Doppler ultrasonography modality.
	\item Implement a Doppler ultrasound imaging technique compliant with the CS framework.
	\item Include this modality in the demonstrator and validate it on in-vitro and in-vivo data.
\end{itemize}

\textbf{Notes:} The demonstrator will be presented at the conference DATE 2017 in Lausanne (SwissTech Convention Center, EPFL): Highlighting Electronics for the Internet of Things Era and Wearable and Smart Medical Devices.
A challenge on Doppler ultrasonography will be organized at the 2017 IEEE International Ultrasonics Symposium (IUS) which will take place in September.
Requirements: Knowledges in compressed sensing is a plus. Skills in Matlab and/or Python.

\textbf{Supervisor:} Prof. Jean-Philippe Thiran

\textbf{Assistants:} D. Perdios (dimitris.perdios@epfl.ch) and A. Besson (adrien.besson@epfl.ch)
\end{document}  